\chapter{Introduction}
\begin{quotation}
    “Free Software is software that respects your freedom and the social solidarity of your community. So it’s free as in freedom.”
\end{quotation}

\begin{flushright}
    Richard M Stallman
\end{flushright}


In this chapter, we will start by setting up the pace and give a general introduction as to why we choose this particular project. We first present a brief background study in which we shall give the reader an insight into our motivation to take up this project followed by the impact our project shall have on various spheres of social and industrial dimensions. We then give an account of similar works wherein we shall compare and contrasts their approach with ours.

We end this chapter by discussing our problem statement and our proposal for the solution.

\section{Background Study}
The advent of technology brought great progress in various avenues of life. It has improved our living standards substantially and made it possible for us to achieve previously unimaginable gains. And needless to say, of past half a century or so, hardware coupled with software has brought infinite strength and weakness at our feet. We have landed on the moon, sent machines to mars, dug ever so deeply into the earth in search of gold and black gold alike and the list goes on. But what really has changed in the past three-four decades is the exponential growth of softwares.

\subsection{Motivation}
Software has brought in massive revolutionary destruction and breakthrough which is giving us hopes to go beyond the imaginative horizons we now inhabit. We are now looking at self-driving cars, cure for cancers, early detection of deadly plagues and also, massive surveillance. And all of this is happening at an alarmingly rapid pace. The code we wrote yesterday is already obsolete today. And the code we are writing today is nothing short of a miracle from yesterday. But that brings us to an important question: Where are we heading towards with it? And are we making sure that such rapid and reckless progress is not coming at the cost of freedom, privacy, and safety?

Our project here is aimed at providing a safe mind and a first step towards being free in this age of massive technological invasion by providing free software alternatives to ubiquitous proprietary softwares.

Free Software Movement\cite{FSF}, started back in 1984, is ever since marching relentlessly, braving all odds put up by those who are reaping unethical profits from the software they write. Our project is aiming to add a bit of our own share towards the movement. Many a time, we are aware that the softwares we are using are proprietary by nature and thus we have no control over it whatsoever. We are forced to accept what comes packed in a pink ribbon without having the leeway to ask for changes or wish for betterment. We are being traded with convenience in return for our privacy because we don’t know what’s going on behind the scenes. So a safer and much better situation would be to resort to free softwares that are developed by communities that we can either directly be part of or can trust them because we can see for ourselves very transparently as to what the communities are doing to and with the software they spawn.

\subsection{Relevance and Social Impact}
The above concerns and issues so raised can be rounded about by a platform like AlterFoss. As software engineers, we are both consumers as well as producers of the software. And in the current world, the role of software and the power it wields is ever so immense and massive respectively. Our code, which is merely a result but also an important proxy of the design that goes behind and many technical and non-technical decisions that were employed during the construction, can now be seen seeking refuge in every domain of our lives. Many a time, it’s part of the time and/or life-critical systems such as medical equipment, self-driving cars and of course, our gadgets. Its omnipresence makes it very potent which then leads to an obvious fork in our lives: how can this power be harvested for the betterment of humanity and not let it go against it?

When our code is hidden from the world, we as developers have all the freedom to employ it as we deem fit. Thus a serious leak of power in the wrong streams. This is how it’s going on so far. Most of the titans of the tech industry follow the proprietary model of software development which essentially means we aren’t allowed to see what’s under the hood. This is rather worrisome.

We the developers have a moral responsibility to not subject our users to such injustice. We shouldn’t misplace their trust. That’s where our project finds most of its relevance. Also, it’s important that early on in our careers (which really start on day 1 of our engineering course), we start contributing to free and open source projects so that we get practical as well as emotional relevance in the tech world.

Our project offers a convenient platform for users to search and find appropriate free and/or open source alternatives to the proprietary softwares they are using. This is helpful to various kind of users. The non-developer community can simply seek alternatives for use whereas developers can run searches on our platform to find projects which they think fit to their skill-set as well as interests.

\subsection{Industrial Impact}
Up to 53\% of companies across industries have switched/are switching towards free and open source software systems and/or models. Within the software industry, as much as 85\% of companies have a project under free and/or open source softwares or are using products of such a model. That’s very encouraging for developers like us who stage a belief in FOSS. For industries too, our platform can be used to find and work on free softwares that they see fit for their uniques circumstances. We may not be reinventing the wheel but we are certainly pulling in a work that can make a difference for all its users. Industries are driven by engineers, and engineers with their work and talent invested in the right places are indisputable.


\section{Related Works}
\begin{itemize}
    \item AlternativeTo\cite{Alternative}: An online web application that gives alternatives to any software indexed by it. It provides any kind of alternative without any restrictions.

          AlternativeTo  is a website which lists alternatives to web-based software, desktop computer software, and mobile apps, and sorts the alternatives by various criteria, including the number of registered users who have clicked the "Like" button for each of them.

          Users can search the site to find better alternatives to an application they are using or previously have used, including free alternatives such as a free web application (cloud computing) which does not require any installation and can be accessed from any browser.

    \item Free Software Directory\cite{FSF}: An online searchable directory maintained by Free Software Foundation. It has over 15,000+ entries of various free softwares. The approach taken here is rather too simple and involves only a search of free softwares whose names are known. It provides you with links to various websites from where you can donwload the said softwares.

\end{itemize}


\section{Problem Statement and Objectives}

\subsection{Problem Statement}
AlterFoss is a web-based application that provides information on the alternatives to proprietary softwares. Users who are now willing to take their privacy in their own hands can make the best use of our platform but all users can do so nevertheless. The platform provides an appropriate and quick response to a user query. As cited already, alternatives are presented to the users on demand.

\subsection{Objectives}
\begin{itemize}
    \item To provide a search for proprietary software and get its corresponding FOSS alternatives.
    \item To allow to like/dislike the alternatives.
    \item To provide an interface to raise a request for alternatives by providing proprietary software details.
    \item To provide an interface for users to suggest alternatives to already present proprietary softwares.
    \item Develop user authentication for the above three tasks.
    \item Develop straightforward and easy to use UI
    \item Develop robust back-end support
    \item Develop a platform that community can use and grow on its own.
\end{itemize}